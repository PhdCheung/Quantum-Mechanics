\documentclass{article}
\usepackage[english]{babel} % 加入英文支持
\usepackage[UTF8, heading=false, scheme=plain]{ctex} % ctex 配置,不影响标题和数字格式
\usepackage{pdfpages}
\usepackage{bm}
\usepackage{listings}
\usepackage{xcolor}
\usepackage{gensymb}
\usepackage{array} % for better columns
\usepackage{amsmath}
\usepackage{amsthm}
\numberwithin{equation}{section}
\usepackage[utf8]{inputenc}
\usepackage{amssymb}
\usepackage{graphicx}
\usepackage{xstring}
\newcommand{\findandreplace}[2]{\StrSubstitute{#1}{\equiv}{:=}[#2]}
\usepackage{dsfont}
\usepackage{fancyhdr} % 添加页眉和页脚的宏包

\author{Albert Cheung}
\title{Introduction to Quantum Mechanics\\ David J. Griffiths}

\begin{document}
	
	\maketitle
	\thispagestyle{empty} % 禁止在封面显示页码


	\newpage
    \thispagestyle{empty} % 禁止在前言显示页码
    \section*{Preface}


	\newpage
	\pagestyle{fancy}
	\fancyhf{} % 清除所有页眉页脚文本
	\fancyhead[L]{\nouppercase{\leftmark}} % 页眉左侧显示章节标题,\nouppercase{} 确保标题是按照小写显示
	\fancyhead[R]{} % 页眉右侧为空
	\renewcommand{\headrulewidth}{0.4pt} % 页眉下方横线的宽度
	\renewcommand{\footrulewidth}{0pt} % 页脚下方横线的宽度(此处设为0pt表示没有横线)
	\fancyfoot[C]{\thepage} % 页脚中间显示页码
	\fancypagestyle{plain}{ % 确保章节的第一页也使用fancy页眉
		\fancyhf{}
		\fancyhead[L]{\nouppercase{\leftmark}}
		\fancyhead[R]{}
		\fancyfoot[C]{\thepage} % plain样式下也显示页码
		\renewcommand{\headrulewidth}{0.4pt}
	}
	
	\tableofcontents
	\thispagestyle{empty} % 禁止在目录页显示页码
	
	\newpage
	\setcounter{page}{1} % 从这页开始页码重置为1

\section{The Wave function}
\subsection{The Schrödinger Equation}

We're looking for the particle's wave function, which can be represented as
\begin{equation}
\Psi(x,t)
\end{equation}
We get it by solving the Schrödinger Equation
\begin{equation}
i\hbar\frac{\partial \Psi}{\partial t}=-\frac{\hbar^2}{2m}\frac{\partial^2\Psi}{\partial x^2}+V\Psi
\end{equation}
where $\hbar$ satisfies
\begin{equation}
	\hbar=\frac{h}{2\pi}=1.054574\times10^{-34}J s
\end{equation}

\subsection{Probability}
In discrete cases, the average value of some function of $j$ is given by
\begin{equation}
	\langle f(j) \rangle=\Sigma_{j=0}^\infty f(j)P(j)
\end{equation}
and we have
\begin{equation}
	\langle j^2\rangle \geq \langle j\rangle
\end{equation}
In continuous cases, we have
\begin{equation}
	P_{ab}=\int_a^b\rho(x)dx
\end{equation}
\begin{equation}
	\int_{-\infty}^{+\infty}\rho(x)dx=1
\end{equation}\begin{equation}
	\langle x\rangle=\int_{-\infty}^{+\infty}x\rho(x)dx
\end{equation}\begin{equation}
	\langle f(x) \rangle=\int_{-\infty}^{+\infty} f(x)\rho(x)dx
\end{equation}\begin{equation}
	\sigma^2 \equiv\langle(\delta x)^2\rangle=\langle x^2\rangle-\langle x\rangle^2
\end{equation}
\subsection{Normalization}
\begin{equation}
	\int_{-\infty}^{+\infty}|\Psi(x,t)|^2dx=1
\end{equation}
and
\begin{equation}
	\frac{d}{dt}\int_{-\infty}^{+\infty}|\Psi(x,t)|^2dx=0
\end{equation}

\subsection{Momentum}
For a particle in state $\Psi$, the expectation value of x is
\begin{equation}
	\langle x\rangle=\int_{-\infty}^{+\infty}x|\Psi(x,t)|^2dx
\end{equation}
Thus
\begin{equation}
	\frac{d\langle x\rangle}{dt}=-\frac{i\hbar}{m}\int\Psi^*\frac{\partial \Psi}{\partial x}dx
\end{equation}
yielding
\begin{equation}
	\langle p\rangle=m\frac{d\langle x\rangle}{dt}=-i\hbar\int(\Psi^*\frac{\partial \Psi}{\partial x})dx
\end{equation}
We have 
\begin{equation}
	\hat{x}=x
\end{equation}
and
\begin{equation}
	\hat{p}=-i\hbar\frac{\partial}{\partial t}
\end{equation}
In general cases, for a function of $Q(x,p)$, we can simply replace every $p$ by $-i\hbar\frac{\partial}{\partial t}$
\begin{equation}
	\langle Q(x,p)\rangle=\int\Psi^*[Q(x,-i\hbar\frac{\partial}{\partial t})]\Psi dx
\end{equation}
\subsection{The Uncertainty Principle}
The wave length of $\Psi$ is related to the momentum of the particle by the de Broglie formula
\begin{equation}
	p=\frac{h}{\lambda}=\frac{2\pi\hbar}{\lambda}
\end{equation}
and we have Heisenberg Uncertainty Principle
\begin{equation}
	\sigma_x\sigma_p\geq\frac{\hbar}{2}
\end{equation}
\newpage
\section{Time-independent Schrödinger Equation}
\subsection{Stationary States}
In this chapter, assuming $V$ is independent of $t$. In that case the Schrodinger equation can be solved by the method of separation of variables, which is
\begin{equation}
	\Psi(x)=\psi(x)\varphi(t)
\end{equation}
After calculating, yielding
\begin{equation}
	\varphi(t)=Ce^{-i\frac{E}{\hbar}t}
\end{equation}
and
\begin{equation}
	-\frac{\hbar^2}{2m}\frac{d^2\psi}{dx^2}+V\psi=E\psi
\end{equation}
The second one is called the time-independent Schrodinger equation (Eq. 2.3).\\
So the wave function can be represented as
\begin{equation}
	\Psi(x,t)=C\psi(x)e^{-i\frac{E}{\hbar}t}
\end{equation}
\\
\\
Reasons of what's so great about separable solutions:\\
\begin{enumerate}
	\item They are stationary states. The probability density 
\begin{equation}
	|\Psi(x,t)|^2=\Psi(x)*\Psi(x)=|\psi(x)|^2
\end{equation}
is time independent. So same thing happens in calculating the expectation value of any dynamical cariable
\begin{equation}
	\langle Q(x,p)\rangle=\int\psi^*Q(x,\frac{\hbar}{i}\frac{d}{dx})\psi dx
\end{equation}
Every expectation value is constant in time.
	\item The total energy is called the Hamiltonian
\begin{equation}
	H(x,p)=\frac{p^2}{2m}+V(x)
\end{equation}
Hamiltonian operator is
\begin{equation}
	\hat{H}=-\frac{\hbar^2}{2m}\frac{\partial^2}{\partial x^2}+V(x)
\end{equation}
Thus the time independent Schrodinger equation can be written as
\begin{equation}
	\hat{H}\psi=E\psi
\end{equation}
After calculating, the variance of H is
\begin{equation}
	\sigma_H^2=E^2=E^2=0
\end{equation}
So the distribution has zero spread. The every measurement of total energy is certain to return the same value $E$.
	\item The general solution is the linear combination of separable solutions. Eq. 2.3 yields an infinite collection of solutions. Any linear combination of solutions is itself a solution. Thus
\begin{equation}
	\Psi(x,t)=\Sigma_{n=1}^{+\infty}c_m\psi_ne^{-i\frac{E_n}{\hbar}t}
\end{equation}
\end{enumerate}
Solve Schrodinger equation yield an infinite set of solutions in general. Thus the wave function is
\begin{equation}
	\Psi(x,t)=\Sigma_{n+1}^{+\infty}c_n\psi_n(x,t)e^{-i\frac{E_n}{\hbar}t}=\Sigma_{n+1}^{+\infty}c_n\Psi_n(x,t)
\end{equation}
Noted that the energies are different, for different stationary states, and the exponentials do not cancel, when you construct $|\Psi|^2$.
\\
\\
Assuming $V(X)$ is time independent, rearrange the Schrodinger equation as
\begin{equation}
	\frac{d^2 \psi(x)}{d x^2}=\frac{2m}{\hbar^2}(V(x)-E)\psi(x)
\end{equation}
then 
\begin{enumerate}
	\item If $V(x)>E\Rightarrow \psi(x)$ curves away from $x$-axis
	\item If $V(x)<E\Rightarrow \psi(x)$ curves toward $x$-axis\\\\
\end{enumerate}
In general, $V(x)$ is finite, $\frac{\partial}{\partial x}$ is continuous, $\frac{\partial^2}{\partial x}$ may be discontinuous (because of time-independent Schrodinger equation).
\subsection{The Infinite Square Well}
The potential energy of an infinite square well is as
\begin{equation}
	V(x)=\left\{\begin{matrix}0,\text{       }0\leq x\leq a\\\infty,\text{       }\text{otherwise}\end{matrix}\right.
\end{equation}
Out of the box, $V(x)=+\infty$, $E$ is finite, thus
	$$V(x)\psi(x)=E\psi(x)$$
	$$\Rightarrow\psi(x)=0$$
In the box, $V(x)=0$, thus the time-independent Schrodinger equation reads
\begin{equation}
	-\frac{\hbar^2}{2m} \frac{d^2\psi}{dx^2}=E\psi
\end{equation}
Rearrange it, yielding
\begin{equation}
	\frac{d^2\psi}{dx^2}+\omega^2\psi=0
\end{equation}
where $\omega\equiv\frac{\sqrt{2mE}}{\hbar}$. The Eq.(2.16) is called the classical simple harmonic oscillator, the general solution is
\begin{equation}
	\psi(x)=A\sin \omega x+B\cos \omega x
\end{equation}
Continuity of $\psi$ requires that
\begin{equation}
	\psi(0)=\psi(a)=0
\end{equation}
so
\begin{equation}
	\omega_n=\frac{n\pi}{a}
\end{equation}
and
\begin{equation}
	E_n=\frac{\hbar^2\omega^2}{2m}=\frac{n^2\pi^2\hbar^2}{2ma^2}
\end{equation}
Normalize it, yielding
\begin{equation}
	\psi_n(x)=\sqrt{\frac{2}{a}}\sin(\frac{n\pi}{a}x)
\end{equation}
\\
\\
$\psi_1$ carries the lowest energy, is called the ground state, the others, whose energies increase in proportion to $n^2$, are called excited states. In the meantime, they are orthogonal, in the sense that
\begin{equation}
	\int\psi_m(x)^*\psi_n(x)dx=0,\text{  }iff\text{  }m\neq n
\end{equation}
Combine orthogonality and normalization into a single statement
\begin{equation}
	\int\psi_m^*\psi_n(x)dx=\delta_{mn}
\end{equation}
where $\delta_{mn}$ is called Kronecker delta, which is defined by
\begin{equation}
	\delta_{mn}=\begin{cases}
		0,m\neq n,\\
		1,m=n.
	\end{cases}
\end{equation}
\\
\\
The functions are complete, which means any other function $\psi(x)$ can be expressed as a linear combination of them
\begin{equation}
	f(x)=\Sigma_{n=1}^{\infty}c_n\psi_n(x)=\sqrt{\frac{2}{a}}\Sigma_{n=1}^{\infty}c_n\sin(\frac{nx}{a}x)
\end{equation}
where
\begin{equation}
	c_n=\int\psi_n(x)^*f(x)dx
\end{equation}
\\
\\
The stationary states of the infinite square well are
\begin{equation}
	\Psi(x,t)=\sqrt{\frac 2a}\sin(\frac{n\pi}{a}x)e^{-iEt}
\end{equation}
where $E$ is from Eq.(2.20). The general solution to he time-independent Schrodinger equation is its linear combination
\begin{equation}
	\Psi(x,t)=\sqrt{\frac 2a}\Sigma_{n=1}^{\infty}c_n\sin(\frac{n\pi}{a}x)e^{-iE_nt}
\end{equation}
where
\begin{equation}
	c_n=\sqrt{2a}\int_0^a\sin(\frac{n\pi}{a}x)\Psi(x,t=0)dx.
\end{equation}
\\
\\
Actually, what $|c_n|^2$tellls you is the probability that a measurement of energy will result in $E_n$, and it was normalized
\begin{equation}
	\Sigma_{n=1}^\infty |c_n|^2\equiv 1
\end{equation}
so
\begin{equation}
	\langle H\rangle=\Sigma |c_n|^2 E_n
\end{equation}
\subsection{The Harmonic Oscillator}





\end{document}






